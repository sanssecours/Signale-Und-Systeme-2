%!TEX TS-program = xelatex

% -- Document Class -----------------------------------------------------------

\documentclass[a4paper, 12pt]{article}

% -- Packages -----------------------------------------------------------------

% Language
\usepackage{polyglossia}
    \setmainlanguage{german}
    \setotherlanguage{english}
% Context sensitive quotation
\usepackage{csquotes}
% Extend options for positioning floats
\usepackage{float}
% Support highlighting of certain parts of the text
\usepackage{framed}
% Customize page properties
\usepackage[top=2cm, bottom=3cm]{geometry}
% Definitions for mathematical type setting
\usepackage{mathtools}
% Headers & Footers
\usepackage[automark, nouppercase]{scrpage2}
% To change the format of titles
\usepackage{titlesec}
% Support for unicode math fonts
\usepackage{unicode-math}
% Extended color support
\usepackage{xcolor}
% Extras for XƎTEX
\usepackage{xltxtra}
% Hyperlinks and pdf properties
\usepackage{hyperref}

% -- Color Definitions --------------------------------------------------------

% Background color for syntax highlighting
\definecolor{bgcolor}{rgb}      {1,     1,      1}

% Custom color definitions
\definecolor{aqua}{rgb}         {0,     0.56,   1}
\definecolor{bluegray}{rgb}     {0.22,  0.46,   0.84}
\definecolor{grape}{rgb}        {0.56,  0,      1}
\definecolor{orchid}{rgb}       {0.41,  0.13,   0.55}
\definecolor{orange}{rgb}       {1,     0.54,   0}
\definecolor{silver}{rgb}       {0.57,  0.57,   0.57}
\definecolor{turquoise}{rgb}    {0,     0.86,   0.84}

% -- Macros -------------------------------------------------------------------

\newcommand{\Title}{Zusätzliche Formeln}
\newcommand{\TitleDescription}{Signale Und Systeme 2}
\newcommand{\Version}{0}
\newcommand{\Subject}{Diverse Formeln für die LVA Signale und Systeme 2}
\newcommand{\KeyWords}{Formeln, Signale, Systeme}

\newcommand{\AuthorOne}{René Schwaiger}
\newcommand{\MailOne}{\href{mailto:sanssecours@f-m.fm}{sanssecours@f-m.fm}}

% -- Document Properties ------------------------------------------------------

% No indendation after paragraph
\setlength\parindent{0cm}

% Hyperref properties
\hypersetup
{
    pdftitle    = {\Title},
    pdfsubject  = {\Subject},
    pdfauthor   = {\AuthorOne},
    pdfkeywords = {\KeyWords},
    colorlinks  = true,
    linkcolor   = black,
    anchorcolor = black,
    citecolor   = silver,
    urlcolor    = orange
}

% -- Fonts --------------------------------------------------------------------

% Use same size for numbers and other text
\defaultfontfeatures{Numbers=Lining}

% Set fonts for document
\setmainfont[Mapping=tex-text]{Avenir Next}
\setsansfont[Mapping=tex-text]{Ubuntu}
\setmonofont[Scale=MatchLowercase]{Menlo}
\setmathfont{Asana-Math.otf}

% Define font styles
\newfontfamily\Zapfino{Zapfino}

% -- Header And Footers -------------------------------------------------------

% Use normal font instead of italic font for head
\renewcommand{\headfont}{\normalfont}

% Set headers and footers
\ihead{\headmark}
\ohead{}
\ifoot{\LeftFooter}
\ofoot{\thepage}

% Set height of head
\setlength{\headheight}{1.8\baselineskip}

% Set thickness of separation line in header, footer
\setheadsepline{0.5pt}
\setfootsepline{0.5pt}

\begin{document}

% -- Section & Paragraph Style ------------------------------------------------

% Set format for section
\titleformat{\section}
    {\large\sffamily\bfseries}  % Large, bold, sans serif font for section
    {}                          % No format applied to whole title
    {0pt}                       % No separation between label and title
    {\thesection~·~}            % Start with section number
    [{\color{orchid}\hrule}]      % Underline with blue ruler

% Set format for other sections and paragraphs
% Color = orchid, Font = bold, sans serif
\titleformat*{\subsection}{\color{orchid}\sffamily\bfseries}
\titleformat*{\subsubsection}{\color{orchid}\sffamily\bfseries}
\titleformat*{\paragraph}{\color{orchid}\sffamily\bfseries}
\titleformat*{\subparagraph}{\color{orchid}\sffamily\bfseries}

\pagestyle{empty}

% -- Text ---------------------------------------------------------------------

{\Large\Title~•~\TitleDescription}
\vskip 0.1cm
{\color{aqua}\hrule}
\vskip 0.2cm

\section{Formeln}

\begin{align*}
    ∑_{k=0}^{N-1} x^k &= \frac{1-x^N}{1-x}\\
    cos² α            &= \frac{1}{2} + \frac{1}{2} · cos 2α\\
    cos α             &= \frac{1}{2} ( e^{jα} + e^{-jα})\\
    sin(z+w)          &= sin(z) · cos(w) + cos(z) · sin(w)\\
    cos(z+w)          &= cos(z) · cos(w) - sin(z) · sin(w)
\end{align*}

\section{Hilfestellungen}

\subsection{Allgemein}

\begin{itemize}
    \item Das Koeffizientenspektrum eines reelwertigen Signals ist symmetrisch um
          $\frac{n}{2}$
    \item $x[n] = x[N-n]$ (reell, gerade) $⇒c_k$ reell, gerade
    \item $x[n] = x[N-n]$ (reell, ungerade) $⇒c_k$ imaginär, ungerade
    \item Periodische Signal $⇔$ Diskretes Spektrum
    \item Nicht periodische Signale $⇔$ Kontinuierliches Spektrum
\end{itemize}

\subsection{Signale}
Unterscheidungen von Signalen durch: \textit{Linearität, Kausalität, Stabilität, Zeitinvarianz}
\subsubsection{Linear}
Ein System ist Linear wenn die Summe von Eingangssignalen gleich die Summe von Einzelantworten $ ∑ y[n] == ∑ x[n] $

\subsubsection{Kausal}
Kausalität kann daran erkannt werden, dass:
\begin{itemize}
  \item Impulsantwort ($h[n]$): $h[n]=0$ $\forall n<0$
  \item Übertragungsfunktion ($H(e^{j\theta})$): Rücktransformieren in $h[n]$, kontrolle siehe oben
  \item Ausgangssignal ($y[n]$): kein $n+1$ in $y[n]$ vorhanden
\end{itemize}

\subsubsection{Stabil}
BIBO stabil, wenn die Summe der Impulsantwort nicht gegen $\infty$ konvergiert
$\sum_{k=-\infty}^{\infty} |h[k]| < \infty$

\subsubsection{Zeitinvariant}
wenn das (vorab vereinfachte) Signal nicht mehr von \textit{t}, bzw. \textit{n} abhängt. Wie z.B.:
\begin{itemize}
  \item $z[n] = z[n+1]$ ... Zeitinvariant
  \item $z[n] = z[n+1]*n$ ... \textbf{nicht} Zeitinvariant
\end{itemize}

\subsection{Fouriertransformation}

\subsubsection{Reelwertige Signale}

\begin{itemize}
    \item Realteil und Betrag der Fouriertransformation sind gerade Funktionen
    \item Imaginärteil und Phase der Fouriertransformation sind ungerade Funktionen
    \item Es genügt die Bestimmung der Fouriertransformation für $Θ ∈ [0,π]$
\end{itemize}

\subsection{Zeit/Frequenzbereich}

\begin{tabular}{lllll}
    Zeitbereich &
    $\color{orange} R_g$  &
    $\color{turquoise} R_u$  &
    $\color{bluegray} jI_g$ &
    $jI_u$\\

    Frequenzbereich &
    $\color{orange} R_g$  &
    $R_u$  &
    $\color{bluegray}jI_g$ &
    $\color{turquoise}jI_u$\\
\end{tabular}

\end{document}
