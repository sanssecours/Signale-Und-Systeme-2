%!TEX TS-program = xelatex

% -- Document Class -----------------------------------------------------------

\documentclass[a4paper, 12pt]{article}

% -- Packages -----------------------------------------------------------------

% Language
\usepackage{polyglossia}
    \setmainlanguage{german}
    \setotherlanguage{english}
% Context sensitive quotation
\usepackage{csquotes}
% Extend options for positioning floats
\usepackage{float}
% Support highlighting of certain parts of the text
\usepackage{framed}
% Customize page properties
\usepackage[top=2cm, bottom=2cm]{geometry}
% Definitions for mathematical type setting
\usepackage{mathtools}
% Headers & Footers
\usepackage[automark, nouppercase]{scrpage2}
% To change the format of titles
\usepackage{titlesec}
% Support for unicode math fonts
\usepackage{unicode-math}
% Extended color support
\usepackage{xcolor}
% Extras for XƎTEX
\usepackage{xltxtra}
% Hyperlinks and pdf properties
\usepackage{hyperref}

% -- Color Definitions --------------------------------------------------------

% Background color for syntax highlighting
\definecolor{bgcolor}{rgb}      {1,     1,      1}

% Custom color definitions
\definecolor{aqua}{rgb}         {0,     0.56,   1}
\definecolor{bluegray}{rgb}     {0.22,  0.46,   0.84}
\definecolor{grape}{rgb}        {0.56,  0,      1}
\definecolor{orchid}{rgb}       {0.41,  0.13,   0.55}
\definecolor{orange}{rgb}       {1,     0.54,   0}
\definecolor{silver}{rgb}       {0.57,  0.57,   0.57}
\definecolor{turquoise}{rgb}    {0,     0.86,   0.84}

% -- Macros -------------------------------------------------------------------

\newcommand{\Title}{Zusätzliche Formeln}
\newcommand{\TitleDescription}{Signale Und Systeme 2}
\newcommand{\Version}{0}
\newcommand{\Subject}{Diverse Formeln für die LVA Signale und Systeme 2}
\newcommand{\KeyWords}{Formeln, Signale, Systeme}

\newcommand{\AuthorOne}{René Schwaiger}
\newcommand{\MailOne}{\href{mailto:sanssecours@f-m.fm}{sanssecours@f-m.fm}}

% -- Document Properties ------------------------------------------------------

% No indendation after paragraph
\setlength\parindent{0cm}

% Hyperref properties
\hypersetup
{
    pdftitle    = {\Title},
    pdfsubject  = {\Subject},
    pdfauthor   = {\AuthorOne},
    pdfkeywords = {\KeyWords},
    colorlinks  = true,
    linkcolor   = black,
    anchorcolor = black,
    citecolor   = silver,
    urlcolor    = orange
}

% -- Fonts --------------------------------------------------------------------

% Use same size for numbers and other text
\defaultfontfeatures{Numbers=Lining}

% Set fonts for document
%\setmainfont[Mapping=tex-text]{Avenir Next}
%\setsansfont[Mapping=tex-text]{Ubuntu}
%\setmonofont[Scale=MatchLowercase]{Menlo}
%\setmathfont{Asana-Math.otf}

% Define font styles
\newfontfamily\Zapfino{Zapfino}

% -- Header And Footers -------------------------------------------------------

% Use normal font instead of italic font for head
\renewcommand{\headfont}{\normalfont}

% Set headers and footers
\ihead{\headmark}
\ohead{}
\ifoot{\LeftFooter}
\ofoot{\thepage}

% Set height of head
\setlength{\headheight}{1.8\baselineskip}

% Set thickness of separation line in header, footer
\setheadsepline{0.5pt}
\setfootsepline{0.5pt}

\begin{document}

% -- Section & Paragraph Style ------------------------------------------------

% Set format for section
\titleformat{\section}
    {\large\sffamily\bfseries}  % Large, bold, sans serif font for section
    {}                          % No format applied to whole title
    {0pt}                       % No separation between label and title
    {\thesection~·~}            % Start with section number
    [{\color{orchid}\hrule}]      % Underline with blue ruler

\pagestyle{empty}

% -- Text ---------------------------------------------------------------------

{\Large\Title~•~\TitleDescription}
\vskip 0.1cm
{\color{aqua}\hrule}
\vskip 0.2cm

\section{Formeln}

\subsection{Summenformeln von Reihen}

\begin{align*}
    ∑_{n=0}^{N-1} x^n &=
        \begin{cases}
            N                 & \text{ if } x = 1\\
            \frac{1-x^N}{1-x} & \text{ otherwise }
        \end{cases}\\
    ∑_{n=0}^{∞} x^n &= \frac{1}{1-x} \quad\text{ if } |x| < 1
\end{align*}

\subsection{Winkelfunktionen}

\begin{alignat*}{3}
    e^{jα}      &= \cos{α} + j\sin{α} \quad&&
    e^{-jα}     &= \cos{α} - j\sin{α}\\
    \cos{α}     &= \frac{e^{jα} + e^{-jα}}{2} \quad&&
    \sin{α}     &= \frac{e^{jα} - e^{-jα}}{2·j}\\
    \cos²{α}    &= \frac{1}{2} + \frac{1}{2} · \cos{2α}\\
    \sin{(z\pm w)} &= \sin{z} · \cos{w} \pm \cos{z} · \sin{w}\\
    \cos{(z\pm w)} &= \cos{z} · \cos{w} \mp \sin{z} · \sin{w}
\end{alignat*}

\section{Hilfestellungen}

\subsection{Allgemein}

\begin{itemize}
    \item Periodisches Signal $⇔$ Diskretes Spektrum
    \item Nicht periodisches Signal $⇔$ Kontinuierliches Spektrum
\end{itemize}

\subsection{Signale}

Unterscheidungen von Signalen durch: \textit{Linearität, Kausalität, Stabilität, Zeitinvarianz}

\subsubsection{Linearität}
Ein System ist linear, wenn sich aus der gewichteten Summe von mehreren Signalen am Eingang, die gewichtete Summe der Einzelantworten der Signale am Ausgang ergibt:
\[
    x[n] = \alpha_1 x₁[n] + \alpha_2 x₂[n] ⇒ y[n] = \alpha_1 y₁[n] + \alpha_2 y₂[n]
\]

\subsubsection{Kausalität}

Kausalität eines LTI-Systems kann daran erkannt werden, dass:
\begin{itemize}
  \item die \emph{Impulsantwort} für negative Werte von $n$ gleich Null ist:
  \[
          h[n]=0 \quad  ∀ n < 0
  \]
  Ist nur die Übertragungsfunktion $H(e^{j\theta})$ gegeben kann man die Impulsantwort kontrollieren, indem man die Übertragungsfunktion durch die inverse Fouriertransformation in die Impulsantwort umwandelt.
  \item im Ausgangssignal $y[n]$ nicht auf Werte des Eingangssignals $x[n+a]$ die mit $a > 0$ die „in der Zukunft” liegen zurückgegriffen wird.
\end{itemize}

\subsubsection{Stabilität}

Ein System ist BIBO stabil, Impulsantwort absolut summierbar ist:
\[
    \sum_{k=-\infty}^{\infty} \left| h[k] \right| < \infty
\]

\subsubsection{Zeitinvarianz}

Ein System $\mathcal{S}$ ist zeitinvariant, wenn ein um ein beliebiges $n_0$ verschobenes Eingangssignal $x[n-n_0]$ ein um $n_0$ verschobenes Ausgangssignal $y[n-n_0]$ liefert.

\[
y[n-n_0] = \mathcal{S} \left[ x[n-n_0]\right] \quad \forall n_0
\]
Ist das System zudem linear, kann man eine Impulsantwort~$h[n]$ angeben, die nur von $n$ abhängt.
Beispiele:
\begin{itemize}
  \item $h[n] = 2^{-n} \sigma[n]$ ... zeitinvariant
  \item $h[n, n'] = \cos(n')~2^{-n} \sigma[n]$ ... \emph{nicht} zeitinvariant
\end{itemize}

\subsection{Fourierreihen}

\begin{itemize}
    \item Das Koeffizientenspektrum eines reellwertigen Signals ist symmetrisch um $\frac{n}{2}$
    \item $x[n] = x[N-n]$ (reell, gerade) $⇒c_k$ reell, gerade
    \item $x[n] = x[N-n]$ (reell, ungerade) $⇒c_k$ imaginär, ungerade
\end{itemize}

\subsection{Fouriertransformation}

\begin{itemize}
    \item Faltung im Zeitbereich $⇔$ Multiplikation im Frequenzbereich
    \item Multiplikation im Zeitbereich $⇔$ Faltung im Frequenzbereich
\end{itemize}

\subsubsection{Reellwertige Signale}

\begin{itemize}
    \item Realteil und Betrag der Fouriertransformation sind gerade Funktionen
    \item Imaginärteil und Phase der Fouriertransformation sind ungerade Funktionen
    \item Es genügt die Bestimmung der Fouriertransformation für $Θ ∈ [0,π]$
\end{itemize}

\subsection{Zeit/Frequenzbereich}

\begin{tabular}{lllll}
    Zeitbereich &
    $\color{orange} R_g$  &
    $\color{turquoise} R_u$  &
    $\color{bluegray} jI_g$ &
    $jI_u$\\

    Frequenzbereich &
    $\color{orange} R_g$  &
    $R_u$  &
    $\color{bluegray}jI_g$ &
    $\color{turquoise}jI_u$\\
\end{tabular}

\end{document}
